%% LyX 1.6.5 created this file.  For more info, see http://www.lyx.org/.
%% Do not edit unless you really know what you are doing.
\documentclass[10pt]{article}
%\usepackage[T1]{fontenc}
%\usepackage[latin9]{inputenc}
\usepackage[letterpaper]{geometry}
\geometry{verbose,tmargin=1in,bmargin=1in,lmargin=1in,rmargin=1in}
%\usepackage{setspace}
\usepackage[dot,sectcntreset]{bibtopic}
\usepackage{fontspec}
\usepackage{setspace}
%\setmainfont{Trebuchet MS}
%\setmainfont[SizeFeatures={Size=10}]{Arial}
\setmainfont[]{Arial}
\usepackage[parfill]{parskip}
%\usepackage{lipsum}
\usepackage{tabularx}
\usepackage[hidelinks]{hyperref}
\usepackage{color}
\usepackage{ragged2e}
\definecolor{richgreen}{RGB}{115,147,61}
\definecolor{richorange}{RGB}{245,128,37}
\definecolor{richblue}{RGB}{25,57,138}
\definecolor{richred}{RGB}{215,89,32}

\usepackage{soul}

\makeatletter
%%%%%%%%%%%%%%%%%%%%%%%%%%%%%% Textclass specific LaTeX commands.
\newenvironment{lyxlist}[1]
{\begin{list}{}
{\settowidth{\labelwidth}{#1}
 \setlength{\leftmargin}{\labelwidth}
 \addtolength{\leftmargin}{\labelsep}
 \renewcommand{\makelabel}[1]{##1\hfill}}}
{\end{list}}

\makeatother
%\usepackage{pslatex}
\usepackage[english]{babel}
\usepackage[absolute]{textpos}

% \usepackage{fancyhdr}

% \pagestyle{fancy}

% \fancyhf{}
% \rhead{\emph{R. Cameron Craddock}}
% \renewcommand{\headrulewidth}{0pt}

\newcommand{\Name}[1]{\textbf{\Large #1}}
\newcommand{\Address}[1]{\medskip #1}
\newcommand{\DMSHeader}[1]{\textbf{#1}\smallskip}

%\newcommand{\Name}[1]{\fontsize{12pt}{14pt}\selectfont\textbf{#1}}
%\newcommand{\Address}[1]{\medskip\fontsize{11pt}{12pt}\selectfont #1}
%\newcommand{\DMSHeader}[1]{\fontsize{12pt}{14pt}\selectfont\textbf{#1}\smallskip}

% adjusting spacing
%\setlength{\lineskip}{0pt}
%\setlength{\parskip}{0pt}
%\setlength{\parsep}{0pt}
%\setlength{\headsep}{0pt}
%\setlength{\topskip}{0pt}
%\setlength{\topmargin}{0pt}
%\setlength{\topsep}{0pt}
%\setlength{\partopsep}{0pt}
%\setlength{\itemsep}{10pt}
%\linespread{1}

\begin{document}
%\setstretch{0.8}
\raggedright
% \thispagestyle{plain}

\begin{center}
  \Name{Richard Cameron Craddock, Ph.D.}\\
   \Address{Department of Diagnostic Medicine\\
   The University of Texas at Austin Dell Medical School}\\
   \Address{9208 Ovalla Drive\\
   Austin, Texas 78749\\
   (404) 625-4973\\
   cameron.craddock@austin.utexas.edu}\\
 \smallskip{}
Prepared \today\\
\end{center}

\bigskip{}
\DMSHeader{Education}

Ph.D., Electrical and Computer Engineering \hfill {05/2003 \textendash{} 12/2009}\\
Georgia Institute of Technology, Atlanta, GA\\
Thesis Title: Support Vector Classification Analysis of Resting State Functional Connectivity fMRI \\
Advisors: Xiaoping Hu, PhD and Helen Mayberg, MD

M.S., Electrical and Computer Engineering \hfill {01/2001 \textendash{} 05/2003} \\
Georgia Institute of Technology, Atlanta, GA

Bachelor of Computer Engineering \hfill {06/1995 \textendash{} 08/1999} \\
Georgia Institute of Technology, Atlanta, GA

\DMSHeader{Postdoctoral Training}

Postdoctoral Fellow \hfill {10/2010 \textendash{} 06/2012} \\
Supervisor: Stephen M. LaConte, PhD\\
Computational Psychiatry Unit, Virginia Tech Carilion Research Institute, Roanoke, VA

Postdoctoral Fellow  \hfill {10/2009 \textendash{} 10/2010} \\
Supervisor: Stephen M. LaConte, PhD\\
Department of Neuroscience, Baylor College of Medicine, Houston, TX

\DMSHeader{Academic and Leadership Appointments}

Associate Professor \hfill {09/2017 \textendash{} present} \\
Department of Diagnostic Medicine, The University of Texas at Austin Dell Medical School, Austin, TX

Research Professor (courtesy appointment) \hfill {05/2016 \textendash{} present} \\
Department of Computer Science and Engineering, NYU Tandon School of Engineering, Brooklyn, NY

Director, Computational Neuroimaging Lab \hfill {08/2014 \textendash{} 08/2017} \\
Center for Biomedical Imaging and Neuromodulation, Nathan S. Kline Institute for Psychiatric Research, Orangeburg, NY

Director of Imaging \hfill {07/2012 \textendash{} 08/2017} \\
Center for the Developing Brain and the Healthy Brain Network, Child Mind Institute, Inc., New York, NY

Research Scientist VI \hfill {07/2012 \textendash{} 11/2015} \\
Research Foundation for Mental Health, Inc., Nathan S. Kline Institute for Psychiatric Research, Orangeburg, NY

\DMSHeader{Employment Activities}

Supervising Research Specialist \hfill {01/2007 \textendash{} 10/2009} \\
Department of Psychiatry and Behavioral Sciences, Emory University School of Medicine, Atlanta, GA

Guest Researcher \hfill {06/2004 \textendash{} 01/2007} \\
Centers for Disease Control and Prevention, Atlanta, GA

Research Assistant \hfill {11/2001 \textendash{} 05/2004} \\
School of Electrical and Computer Engineering, Georgia Institute of Technology, Atlanta, GA

Engineer \hfill {01/2000 \textendash{} 09/2001} \\
Opuswave Networks Inc., Colorado Springs, CO

Junior Systems Engineer \hfill {08/1998 \textendash{} 12/1998} \\
Lucent Technologies Product Realization Center, Atlanta, GA

Stokes Undergraduate Scholar \hfill {08/1995 \textendash{} 12/1999} \\
Central Intelligence Agency, Washington DC

\DMSHeader{Honors and Awards}

\noindent Biobehavioral Research Awards for Innovative New Scientists, NIH/NIMH \hfill 2013

\noindent Poster Award, 1\textsuperscript{st} Place Functional Imaging, ISMRM, Montreal \hfill 2011

\noindent NARSAD Young Investigator Award, BBRF \hfill 2010

\noindent Philips Travel Stipend Award, Resting State Workshop \hfill 2009

\noindent Organization of Human Brain Mapping Trainee Abstract Award \hfill 2009

\noindent CCB/IPAM Mathematics in Brain Imaging Summer Fellow \hfill 2008

\noindent ISMRM Educational Stipend \hfill 2006, 2007, 2008

\noindent BioInfoSummer 2006 Travel Scholarship \hfill 2006

\noindent Georgia Tech Office of Minority Education Tower Award \hfill 2003

\noindent Central Intelligence Agency Meritorious Unit Citation \hfill 1998

\noindent Stokes Undergraduate Scholarship \hfill 1995 \textendash{} 1999

\DMSHeader{Professional Memberships and Activities with Leadership Positions}

The Organization for Human Brain Mapping \hfill 2006 \textendash{} present\\
Education Chair \hfill (2018 \textendash{} 2020)\\
Chair, Education Committee \hfill (2019)\\
Member, Program Committee \hfill (2018 \textendash{} 2020)\\
Founding chair, Open Science Special Interest Group \hfill (2016 \textendash{} 2017)\\
Co-organizer, Brain Art Exhibition \hfill (2011 \textendash{} 2014)

\DMSHeader{Educational Activities}

{\addtolength{\leftskip}{.2 in}
\noindent \DMSHeader{Teaching Activities}

Performing High-Throughput Connectomes Analysis with C-PAC \\
Role: organizer and primary lecturer \\
University Southern California, Pasadena, CA, 1-day course, 5 students \hfill 02/2019\\
UNAM, Quer\'{e}taro, MX, 2-day course, \approx 25 students \hfill 10/2016\\
Child Mind Institute, New York, NY, 2-day course, \approx 50 students \hfill 12/2015\\
UNAM, Quer\'{e}taro, MX, 2-day course, \approx 40 students \hfill 10/2015

PSY 394U Methods for fMRI, Austin, TX, 20 students, \hfill 10/2018 \\
Datasharing Repositories for Neuroimaging Research \\
Role: Guest Lecturer

2018 Australian Connectomics School, Melbourne, AU, 60 students, \hfill 05/2018\\
Building Functional Connectomes  \\
Role: Lecturer 

Brain Imaging Data Standard Short Course, Austin, Texas, 20 students \hfill 07/2018 \\
Role: Organizer and lecturer

Functional Neuroimaging: A Hands On Approach \\
Role: taught several lectures of 4-day course on topics including acquiring fMRI data, fMRI preprocessing, and resting state fMRI data analysis\\
Brain Institute, PUCRS, Porto Alegre, BR, \approx 60 students, \hfill 07/2015\\
Brain Institute, PUCRS, Porto Alegre, BR, \approx 60 students, \hfill 08/2014\\
Brain Institute, PUCRS, Porto Alegre, BR, \approx 60 students, \hfill 08/2013

EE 0407\textendash03, PUCRS Porto Alegre, BR, \approx 50 students, \hfill 07/2014\\
Biomedical Instrumentation and Medical Imaging --- Mapping the Connectivity fo the Human Brain\\
Role: taught 1 credit of 3 credit course (4 3-hour lectures)\\

Brainhack \hfill 2012 \textendash{} present\\
An innovative educational workshop I developed to bring together researchers from disparate backgrounds to learn from one another and collaborate to develop computational solutions to open problems in Neuroscience. Some events are distributed across many locations over the same weekend, with sites connecting using video conference to share content and collaborate. Brainhack has now become a widely spread phenomena with many regional events being held every year. Below are just those that I have organized.

Brainhack Global 2019, distributed, global organizer \hfill 11/2019 \\
Over 1,000 participants at 30 events in 16 countries across 4 continents

Brainhack Global 2018, distributed, global organizer \hfill 05/2018 \\
Over 1,000 participants at 33 events in 14 countries across 3 continents

Brainhack Global 2017, distributed, global organizer \hfill 03/2017 \\
Over 1,000 participants at 36 events in 16 countries across 4 continents

Brainhack LA 2016, Los Angeles, CA, co-organizer \hfill 11/2016 \\
Big Data Tools for Connectomics, \approx 50 attendees

Brainhack Vienna 2016, Vienna, AT, organizer \hfill 9/2016 \\
Reproducibility and Reliability in Connectomics, \approx 70 attendees

Brainhack MX 2015, Quer\'{e}taro, MX, local organizer \hfill 10/2015 \\
31 attendees

Brainhack Americas 2015, distributed, global organizer \hfill 10/2015 \\
211 attendees across 9 locations in 3 countries

Brainhack Eastern Daylight Time, distributed, global organizer \hfill 10/2014 \\
242 attendees across 8 locations in 4 countries 

Brainhack 2013, Paris, FR, Co-organizer \hfill 10/2013\\
70 attendees

2012 BrainHack and UnConference, Leipzig, DE, Co-organizer \hfill 09/2012\\
72 attendees

} % end teaching text box

\DMSHeader{Advising and Mentoring}

{\addtolength{\leftskip}{.2 in}

\noindent \DMSHeader{Undergraduate and Post Baccalaureate Students}

\noindent Mir Ali, UG Research Assistant, Supv. \hfill 08/2019 \textendash{} present

\noindent Henry Rossiter, UG Research Assistant, Supv. \hfill 01/2019 \textendash{} present

\noindent Manwitha Dodla, PB Research Assistant, Supv. \hfill 01/2018 \textendash{} 05/2019 

\noindent Druv Patel, UG Research Assistant, Supv. \hfill 08/2018 \textendash{} present

\noindent Faez Baqai, Health Leadership Apprentice, Co-Supv. \hfill 02/2018 \textendash{} 08/2018 \\
Undergraduate student, Biomedical Engineering, UT, Austin, TX

\noindent Swanand Rakhade, Health Leadership Apprentice, Supv. \hfill 02/2018 \textendash{} 05/2018 \\
Undergraduate student, Computational Biology, UT, Austin, TX

\noindent Alisa Omelchencko, UG Research Assistant, Supv. \hfill 09/2016 \textendash{} 06/2018 \\
Laboratory Technician, Landos Biopharma, Inc., New York, NY

\noindent David O'Connor, PB Research Assistant, Supv. \hfill 10/2013 \textendash{} 05/2017 \\
Ph.D. student, Biomedical Engineering, Yale University, New Haven, CT

\noindent Amalia McDonald, PB Research Assistant, Supv. \hfill 10/2013 \textendash{} 08/2016 \\
Ph.D. student, Psychology, University of Virginia, Charlottesville, VA

\noindent Benjamin Puccio, PB Research Assistant, Supv. \hfill 01/2016 \textendash{} 08/2016 \\
Medical student, Poland

\noindent Zarrar Shehzad, PB Research Assistant, Supv. \hfill 02/2014 \textendash{} 08/2016 \\
Postdoctoral Fellow, Yale University, New Haven, CT


\noindent \DMSHeader{Graduate Students}

\noindent Pu Zhao, M.S. student, UT ECE, Primary advisor \hfill 01/2020 \textendash{} present

\noindent Anibal Solon Heinsfeld, Ph.D. student, UT CS, Primary advisor \hfill 08/2019 \textendash{} present

\noindent Abigail Dowd, Ph.D. student, UT ECE, Primary advisor \hfill 08/2018 \textendash{} 05/2019

\noindent \DMSHeader{Postdoctoral Fellows}

\noindent Timothy Weng, Ph.D., Diagnostic Medicine, UT DMS, Supv. \hfill 10/2018 \textendash{} present

\noindent Chris Foulon, Ph.D., Diagnostic Medicine, UT DMS, Supv. \hfill 10/2018 \textendash{} 08/2019

\noindent James Pooley, Ph.D., Nathan Kline Institute, Supv. \hfill 03/2016 \textendash{} 12/2016\\
Data Analyst, Success Academy Charter Schools, New York, NY

\noindent Jordan Muraskin, Ph.D., Nathan Kline Institute, Supv. \hfill 05/2016 \textendash{} 01/2017\\
Chief Technology Officer, deCervo, New York, NY

\noindent Nicholas Van Dam, Ph.D., Nathan Kline Institute, Co-Supv. \hfill 04/2015 \textendash{} 05/2015\\
Senior Lecturer in Psychological Sciences, University of Melbourne, Melbourne, AU

\noindent \DMSHeader{Dissertation and Thesis Committees}

\noindent Chris Foulon, Ph.D. student, Neuroscience, Paris, FR \hfill 08/2018 \\
Post Doctoral Fellow, Diagnostic Medicine, UT Dell Medical School, Austin, TX

\noindent Anibal Solon, M.S. student, CS PUCRS Porto Alegre, BR \hfill 03/2016 \\
Research Programmer, Child Mind Institute, Inc. New York, NY

} % end mentoring text box

\DMSHeader{Grants}

{\addtolength{\leftskip}{.2 in}

\DMSHeader{Current}

NIMH, 1R24MH114806, Milham and Craddock (MPI) \hfill 09/2018 \textendash{} 05/2021\\
C-PAC: A Configurable, Compute-optimized, Cloud-enabled Neuroimaging Analysis Software For Reproducible Translational And Comparative Neuroscience\\
\$545,319 Total (UT Subcontract)\\
Role: MPI 10\%

JDRF, 3-SRA-2019-759-M-B, Virostko and Powers \hfill 4/1/2019 \textendash{} 3/31/2022\\
Multi-center Assessment of Pancreatic Volume in Type 1 Diabetes\\
\$1,500,000 Total\\
Role: Co-investigator 10\%

\DMSHeader{Pending}

NIMH, R01, Franco and Milham \hfill 09/01/2020 \textendash{} 08/31/2024\\
Real-time fMRI Prediction of Eye Gaze\\
\$150,000 Sub\\
Role: Co-investigator 10\%

\DMSHeader{Completed}

NIMH, 1R01MH101555-S1, Craddock, \hfill 07/2015 \textendash{} 05/2016 \\
Real-time fMRI Neurofeedback Based Stratification of Default Network Regulation (supplement)\\
\$168,357 Total\\
Role: PI 10\%

NIMH, 1R01MH101555, Craddock, \hfill 08/2013 \textendash{} 08/2017\\
Real-time fMRI Neurofeedback Based Stratification of Default Network Regulation\\
\$2,456,896 Total\\
Role: PI 50\%

NIMH Contract, Craddock, \hfill 02/2014 \textendash{} 07/2014 \\
{{C}-{P}{A}{C} integration with the National Database for Autism Research ({N}{D}{A}{R})}\\
\$75,000 Total \\
Role: PI 20\%

BBRF, NARSAD Young Investigator Award \hfill 01/2011 \textendash{} 01/2013 \\
Neuro-Feedback for Default Mode Network Regulation in Major Depressive Disorder\\
\$60,000 Total \\
Role: PI 10\%

\DMSHeader{Grant Review}

NIMH BRAIN R01 \hfill 03/2020

NINDS BRAIN K99 \hfill 08/2019

NINDS BRAIN K99 \hfill 06/2019

NIMH Computational Psychiatry R01 \hfill 03/2019

NIMH ZMH1 ERB-D (04) Treatment Development – Psychosocial Interventions \hfill 02/2019

New Jersey Commission on Brain Injury Research \hfill 2014 \textendash{} 2017 

NSF/NIH Collaborative Research on Computational Neuroscience \hfill 2015

}

\DMSHeader{Other Research Activities}

{\addtolength{\leftskip}{.2 in}

\DMSHeader{Intramural Research}

University of Texas Whole Communities Whole Health\\
Component of the Bridging Barriers initiative that is engaging the community into research about their environment to help them make decisions to improve their health. Involves developing, maintaining, and deploying remote sensors to monitor environmental factors such as air quality and ambient noise along with individual behaviors such as sleep and actigraphy. Community members can monitor their data using a web-based dashboard and are provided suggestions for interventions aimed at improving health outcomes. My role is maintaining a mobile phone based passive and active monitoring system and the dashboard for providing feedback to participants.\\
Current effort: 7\% (funded)

\DMSHeader{Data Sharing}

The Enhanced Rockland Sample Neurofeedback Study Dataset \\
A neuroimaging database of 180 adults (21--45 years old, 50\% female) with a variety of clinical and sub-clinical psychiatric symptoms performing a variety of tasks for assessing the default mode network. \href{http://fcon_1000.projects.nitrc.org/indi/enhanced/}{http://fcon\_1000.projects.nitrc.org/indi/enhanced/}\\
Role: Principle Investigator

Intrinsic Brain Activity Test-Retest Dataset \\
This dataset consists of two ten-minute resting state fMRI scans and two multi source interference task fMRI scans acquired during the same scanning session for thirty-six adults (20--48 years old). 14 of the participants returned for a second scanning session using the same scanning procedures. \href{http://fcon_1000.projects.nitrc.org/indi/CoRR/html/ibatrt.html}{http://fcon\_1000.projects.nitrc.org/indi/CoRR/html/ibatrt.html}\\
Role: Contributor

The Enhanced Rockland Sample Dataset \\
A database of deep phenotyping and a comprehensive connectome neuroimaging assessment on individuals with a variety of clinical and sub-clinical psychiatric symptoms from across the lifespan. \href{http://fcon_1000.projects.nitrc.org/indi/enhanced/}{http://fcon\_1000.projects.nitrc.org/indi/enhanced/}\\
Role: Co-Investigator and Contributor

ADHD-200 Preprocessed Data \\
Preprocessed functional and structural data for 374 children and adolescents who suffer from ADHD and 598 typically developing controls from the ADHD-200 sample. Data was processed using three different software pipelines. \href{http://neurobureau.projects.nitrc.org/ADHD200/Introduction.html}{http://neurobureau.projects.nitrc.org/ADHD200/Introduction.html}\\
Role: Co-Principle Investigator, Contributor

ABIDE Preprocessed Data \\
Preprocessed functional and structural data for 539 individuals suffering from autism and 573 typical controls from the ABIDE dataset. Data was processed using four different functional processing pipelines and three different structural processing pipelines. \href{http://preprocessed-connectomes-project.github.io/abide/}{http://preprocessed-connectomes-project.github.io/abide/}\\
Role: Co-Principle Investigator, Contributor

The Neurofeedback Skull-stripped (NFBS) repository \\
A database of 125 manually skull-stripped T1-weighted anatomical MRI scans from the the Enhanced Rockland Sample Neurofeedback Study. \href{http://preprocessed-connectomes-project.org/NFB_skullstripped/index.html}{http://preprocessed-connectomes-project.org/NFB\_skullstripped/index.html}\\
Role: Principle Investigator, Contributor

The Healthy Brain Network Serial Scanning Initiative \\
A database of 13 adult participants that were repeatedly scanned under each of four scan conditions across 12 sessions. The specific conditions varied respect to level of engagement, and included: 1) resting state, 2) naturalistic viewing of a sequence of abstract shapes, 3) naturalistic viewing of highly engaging movies and 4) performance of an active task. \href{http://fcon_1000.projects.nitrc.org/indi/hbn_ssi/}{http://fcon\_1000.projects.nitrc.org/indi/hbn\_ssi/}\\
Role: Co-Investigator

}

\DMSHeader{Technology Development}

{\addtolength{\leftskip}{.2 in}

\DMSHeader{Patents}

European Patent 11188849.1 \textendash{} 1560 \hfill 11/11/2011 \\
US Patent 20,130,144,154 \hfill 06/2013 \\
Method and apparatus for visualization of tissue perfusion by means of assessing BOLD signal fluctuations\\
Role: Co-inventor

\DMSHeader{Software Applications}

Configurable Pipeline for the Analysis of Connectomes \\
A python-based open source software package for performing connectivity analyses using functional MRI data on high-performance computing architectures.\\
Address: \href{http://fcp-indi.github.io}{http://fcp-indi.github.io}\\
Role: Project Director, Co-Principal Investigator

OpenCogLab Repository \\
Repository of free and open source implementations of computerized experiments for assessing human cognition.\\
Address: \href{http://opencoglabrepository.github.io/}{http://opencoglabrepository.github.io/}\\
Role: Principle Investigator, Contributor

pyClusterROI \\
An open source python library for parcellating functional MRI data using spatially constrained normalized-cut spectral clustering.\\
Address: \href{http://ccraddock.github.io/cluster_roi}{http://ccraddock.github.io/cluster\_roi}\\
Role: Primary Developer

The Preprocessed Connectomes Project Quality Assessment Protocol \\
An open source python library for estimating several different quality measures from functional and structural MRI data.\\
Address: \href{http://preprocessed-connectomes-project.github.io/quality-assessment-protocol/}{http://preprocessed-connectomes-project.github.io/quality-assessment-protocol/}\\
Role: Principle Investigator, Developer

}

\DMSHeader{Publications}

{\addtolength{\leftskip}{.2 in}

\DMSHeader{Peer-reviewed publications}

Google Scholar Statistics - Citations: 9,109, H-index: 40, I10-index: 63 as of \today

\begin{flushleft}
\bibliographystyle{plainyr-rev}
\begin{btSect}{ccraddock_journal_pubs}
\btPrintAll%
\end{btSect}
\end{flushleft}
}

\DMSHeader{Editorial Responsibilities}

GigaScience\\
Editorial Board Member \hfill 2016 \textendash{} present\\
Guest Editor Brainhack Thematic Series \hfill 2016 \textendash{} present

NeuroImage\\
Guest Editor Special Issue on Brain Segmentation and Parcellation \hfill 2018\\
Video Advisor \hfill 2010 \textendash{} 2011

Brainhack Proceedings\\
Editor \hfill 2015 \textendash{} present

NeuroImage, Human Brain Mapping, Journal of Neuroscience Methods, Frontiers in Systems Neuroscience, IEEE Transactions in Medical Imaging, Magnetic Resonance Imaging, Biological Psychiatry, Frontiers in Neuroanatomy, Neuroinformatics, JAMA Psychiatry, Nature Scientific Data, GigaScience \\
Reviewer \hfill 2008 \textendash{} present

17\textsuperscript{th} Meeting of the Organization for Human Brain Mapping (2011), 16\textsuperscript{th} Meeting of the Organization for Human Brain Mapping (2010), 13\textsuperscript{th} International Conference on Medical Image Computing and Computer Assisted Intervention (2010)\\
Conference Abstract Reviewer

\DMSHeader{Invited Presentations, Posters and Abstracts}

{\addtolength{\leftskip}{.2 in}

\DMSHeader{International}

\begin{flushleft}
\bibliographystyle{plainyr-rev}
\begin{btSect}{international_talk_abstracts}
\btPrintAll%
\end{btSect}
\end{flushleft}

\DMSHeader{National}

\begin{flushleft}
\bibliographystyle{plainyr-rev}
\begin{btSect}{national_talk_abstracts}
\btPrintAll%
\end{btSect}
\end{flushleft}

\DMSHeader{Regional}

\begin{flushleft}
\bibliographystyle{plainyr-rev}
\begin{btSect}{regional_talk_abstracts}
\btPrintAll%
\end{btSect}
\end{flushleft}

}

\DMSHeader{Community Service}

Child Mind Institute Endeavor Scientist Program \hfill 2011 \textendash 2014\\
Mentor

Georgia State Science and Engineering Fair, Athens, GA \hfill 2005\\
Judge

Georgia Tech Intel Opportunity Scholars, Atlanta, GA \hfill 2002 \textendash{} 2003\\
Mentor

\DMSHeader{Other Education/Certifications}

\noindent Siemens ICE and SDE programming courses \hfill 2011

\DMSHeader{For More Information:}

\noindent Google Scholar Citations: \href{http://tinyurl.com/CameronCraddockCitations}{http://tinyurl.com/CameronCraddockCitations}

\noindent Impact Story: \href{https://impactstory.org/u/0000-0002-4950-1303}{https://impactstory.org/u/0000-0002-4950-1303}

\noindent Github: \href{https://github.com/ccraddock}{https://github.com/ccraddock}

\noindent ResearchGate: \href{https://www.researchgate.net/profile/Cameron\_Craddock}{https://www.researchgate.net/profile/Cameron\_Craddock}

\noindent Researcher ID:\@\href{http://www.researcherid.com/rid/P-1980-2014}{http://www.researcherid.com/rid/P-1980-2014}

\noindent ORCiD: \href{http://www.researcherid.com/rid/P-1980-2014}{http://orcid.org/0000-0002-4950-1303}

\noindent SciENcv: \href{https://www.ncbi.nlm.nih.gov/myncbi/richard.craddock.1/cv/18275/}{https://www.ncbi.nlm.nih.gov/myncbi/richard.craddock.1/cv/18275/}

\noindent SlideShare: \href{http://www.slideshare.net/CameronCraddock}{http://www.slideshare.net/CameronCraddock}
\end{document}
